\documentclass[english,man]{apa6}

\usepackage{amssymb,amsmath}
\usepackage{ifxetex,ifluatex}
\usepackage{fixltx2e} % provides \textsubscript
\ifnum 0\ifxetex 1\fi\ifluatex 1\fi=0 % if pdftex
  \usepackage[T1]{fontenc}
  \usepackage[utf8]{inputenc}
\else % if luatex or xelatex
  \ifxetex
    \usepackage{mathspec}
    \usepackage{xltxtra,xunicode}
  \else
    \usepackage{fontspec}
  \fi
  \defaultfontfeatures{Mapping=tex-text,Scale=MatchLowercase}
  \newcommand{\euro}{€}
\fi
% use upquote if available, for straight quotes in verbatim environments
\IfFileExists{upquote.sty}{\usepackage{upquote}}{}
% use microtype if available
\IfFileExists{microtype.sty}{\usepackage{microtype}}{}

% Table formatting
\usepackage{longtable, booktabs}
\usepackage{lscape}
% \usepackage[counterclockwise]{rotating}   % Landscape page setup for large tables
\usepackage{multirow}		% Table styling
\usepackage{tabularx}		% Control Column width
\usepackage[flushleft]{threeparttable}	% Allows for three part tables with a specified notes section
\usepackage{threeparttablex}            % Lets threeparttable work with longtable

% Create new environments so endfloat can handle them
% \newenvironment{ltable}
%   {\begin{landscape}\begin{center}\begin{threeparttable}}
%   {\end{threeparttable}\end{center}\end{landscape}}

\newenvironment{lltable}
  {\begin{landscape}\begin{center}\begin{ThreePartTable}}
  {\end{ThreePartTable}\end{center}\end{landscape}}

  \usepackage{ifthen} % Only add declarations when endfloat package is loaded
  \ifthenelse{\equal{\string man}{\string man}}{%
   \DeclareDelayedFloatFlavor{ThreePartTable}{table} % Make endfloat play with longtable
   % \DeclareDelayedFloatFlavor{ltable}{table} % Make endfloat play with lscape
   \DeclareDelayedFloatFlavor{lltable}{table} % Make endfloat play with lscape & longtable
  }{}%



% The following enables adjusting longtable caption width to table width
% Solution found at http://golatex.de/longtable-mit-caption-so-breit-wie-die-tabelle-t15767.html
\makeatletter
\newcommand\LastLTentrywidth{1em}
\newlength\longtablewidth
\setlength{\longtablewidth}{1in}
\newcommand\getlongtablewidth{%
 \begingroup
  \ifcsname LT@\roman{LT@tables}\endcsname
  \global\longtablewidth=0pt
  \renewcommand\LT@entry[2]{\global\advance\longtablewidth by ##2\relax\gdef\LastLTentrywidth{##2}}%
  \@nameuse{LT@\roman{LT@tables}}%
  \fi
\endgroup}


\ifxetex
  \usepackage[setpagesize=false, % page size defined by xetex
              unicode=false, % unicode breaks when used with xetex
              xetex]{hyperref}
\else
  \usepackage[unicode=true]{hyperref}
\fi
\hypersetup{breaklinks=true,
            pdfauthor={},
            pdftitle={When Interactions Bias Corrections: A Missing Data Correction for Centered Predictors},
            colorlinks=true,
            citecolor=blue,
            urlcolor=blue,
            linkcolor=black,
            pdfborder={0 0 0}}
\urlstyle{same}  % don't use monospace font for urls

\setlength{\parindent}{0pt}
%\setlength{\parskip}{0pt plus 0pt minus 0pt}

\setlength{\emergencystretch}{3em}  % prevent overfull lines

\ifxetex
  \usepackage{polyglossia}
  \setmainlanguage{}
\else
  \usepackage[english]{babel}
\fi

% Manuscript styling
\captionsetup{font=singlespacing,justification=justified}
\usepackage{csquotes}
\usepackage{upgreek}

 % Line numbering
  \usepackage{lineno}
  \linenumbers


\usepackage{tikz} % Variable definition to generate author note

% fix for \tightlist problem in pandoc 1.14
\providecommand{\tightlist}{%
  \setlength{\itemsep}{0pt}\setlength{\parskip}{0pt}}

% Essential manuscript parts
  \title{When Interactions Bias Corrections: A Missing Data Correction for
Centered Predictors}

  \shorttitle{Interactions bias correlations}


  \author{Dustin Fife\textsuperscript{1}}

  \def\affdep{{""}}%
  \def\affcity{{""}}%

  \affiliation{
    \vspace{0.5cm}
          \textsuperscript{1} Rowan University  }

 % If no author_note is defined give only author information if available
      \newcounter{author}
                              \authornote{
          Correspondence concerning this article should be addressed to Dustin Fife
          , 201 Mullica Hill Road Glassboro, NJ 08028 .
           E-mail: \href{mailto:fife.dustin@gmail.com}{\nolinkurl{fife.dustin@gmail.com}} 
          }
                    

  \abstract{It is commonly advised to center predictors in multiple regression,
especially in the presence of interactions (J. Cohen, Cohen, West, \&
Aiken, 2013). This will enhance the interpretation of regression
parameters, and (arguably; Dalal \& Zickar, 2012; Echambadi \& Hess,
2007; Kromrey \& Foster-Johnson, 1998) reduce multicollinearity.
However, I demonstrate that with missing data, centering predictors may
bias parameter estimates. I develop a Pearson-Lawyley-based (Aitken,
1935; Lawley, 1944) correction (called \(r_{pl}\)) that is insensitive
to centering, then evaluate its performance via Monte Carlo Simulation.}
  \keywords{missing data, selection, range restriction, interactions \\

    \indent Word count: 1994
  }




  \usepackage{colortbl}

\usepackage{amsthm}
\newtheorem{theorem}{Theorem}
\newtheorem{lemma}{Lemma}
\theoremstyle{definition}
\newtheorem{definition}{Definition}
\newtheorem{corollary}{Corollary}
\newtheorem{proposition}{Proposition}
\theoremstyle{definition}
\newtheorem{example}{Example}
\theoremstyle{remark}
\newtheorem*{remark}{Remark}
\begin{document}

\maketitle

\setcounter{secnumdepth}{0}



Centering predictors often improves the interpretation of coefficients
(J. Cohen et al., 2013). Some have suggested it increases precision by
reducing multicollinearity (J. Cohen et al., 2013). Although this last
advantage is controversial (Dalal \& Zickar, 2012; Echambadi \& Hess,
2007; Kromrey \& Foster-Johnson, 1998), none (that I know of)
recommended against centering (except when the predictors have
meaningful zero; J. Cohen et al., 2013). However, in one situation
centering predictors will exacerbate bias.

In this paper, I show how centering can lead to substantial bias when
data are missing. First, I review the advantages of centering. Next, I
show how centering predictors may change a \enquote{Missing At Random}
into a \enquote{Missing \emph{Not} at Random} situation. Finally, I
introduce a correction that allows researchers to center predictors
without bias, and assess its performance via Monte Carlo Simulation.

\subsection{Regression and Centering}\label{regression-and-centering}

Suppose a university wishes to assess the impact of socioeconomic status
(\(SES\)) on first year GPA (\(FYGPA\)), and wishes to correct for
missing data (let us assume the university selected based on \(SAT\)
scores). Further suppose these two predictors interact (e.g., for those
with high \(SAT\) scores, \(SES\) is more predictive of \(FYGPA\)).
Mathematically,

\[ FYGPA = b_0 + b_1SES + b_2SAT + b_3SES\cdot SAT \] The researcher
might be inclined to center both \(SES\) and \(SAT\) scores. Doing so
supposedly has two advantages. First, the coefficients for the centered
variables are more interpretable (J. Cohen et al., 2013). Recall that
with interactions, the relationship between \(SES\) and \(FYGPA\) is
non-linear; the slope between \(SES\) and \(FYGPA\) changes depending on
the value of \(SAT\). When centered, the \(b_1\) parameter is the
\emph{average} slope of \(FYGPA\) on \(SES\) across all values of
\(SAT\).

The second purported advantage of centering is that is removes
\enquote{nonessential} collinearity (Aiken \& West, 1991; J. Cohen et
al., 2013). The covariance between the interaction variable
(\(SES\cdot SAT\)) and either predictor is a function of the means of
\(SAT\) and \(SES\) (Aiken \& West, 1991, p. 180, Equation A.13):

\begin{equation}
cov(SES, SES\cdot SAT) = s^2_{SES} \overline{SAT} + cov(SAT, SES) \overline{SES}
\label{eq:correction}
\end{equation}

\noindent (The above equation only applies when each predictor is
symmetrical). When both predictors are centered, the means are zero and
\(cov(SES, SES\cdot SAT)\) vanishes. This is what is called
\enquote{nonessential multicollinearity,} or collinearity attributable
to the means of the predictors.

Some (e.g., J. Cohen et al., 2013) argue removing essential
multicollinearity increases the precision of parameter estimates since
multicollinearity tends to inflate standard errors. However, others
(Dalal \& Zickar, 2012; Echambadi \& Hess, 2007; Kromrey \&
Foster-Johnson, 1998) demonstrate precision is unaffected by centering.

Regardless of whether centering affect precision, it is considered wise
practice, at least for its interpretative advantages. However, under
missing data, centering may inflate bias.

\subsection{Interactions and Missing
Data}\label{interactions-and-missing-data}

In concurrent validity designs, when interactions exist, data are
\enquote{Missing Not at Random} (MNAR; Little \& Rubin, 2014; Rubin,
1976), which means the probability of missingness is correlated with
both observable and unobservable data. To understand why, consider our
previous example. Suppose students were selected based on \(SAT\) scores
and the researcher wishes to assess the correlation between
socioeconomic status (\(SES\)) and \(FYGPA\). However, they want to know
the unattenuated correlation, but unfortunately only have incumbent data
for \(SES\). Assuming \(SES\) itself is not a cause of attrition (or
selection), missingness was actually cased by two variables:

\begin{enumerate}
\def\labelenumi{(\arabic{enumi})}
\tightlist
\item
  \(SAT\) scores. Since these were recorded before selection, these data
  are MAR (meaning missingness is caused by observable data).
\item
  \(SAT\cdot SES\) scores. This product term is correlated with the
  probability of missingness such that it is independent of \(SAT\) and
  \(SES\) alone (since an interaction is present). Some of these product
  scores are missing (because they were not selected into the
  university), rending them unobservable. Consequently, these data are
  MNAR.
\end{enumerate}

Notice that the data are MNAR, regardless of whether \(SES\) itself is a
cause of missingness (again, because the product variable is missing for
certain applicants). Had \(SES\) been measured before selection on
\(SAT\) occurred (i.e., in a predictive validity design), the data would
be MAR.

In most situations, the fact that the data are MNAR is not problematic.
One need not actually model the cause of missingness to render a
situation MAR. Rather, one simply needs a correlate of the cause of
missingness (Collins, Schafer, \& Kam, 2001). With uncentered variables,
the correlation between the predictors and their product is high and can
be control for by using applicant \(SAT\) scores. When we center the
variables, however, that correlation vanishes.

\section{Correction Procedure}\label{correction-procedure}

Recall the Pearson-Lawley equation (Aitken, 1935; Lawley, 1944) corrects
estimates when missingness occurs on one or more variables. As a
multivariate extension of Case III, it requires two inputs:

\begin{enumerate}
\def\labelenumi{(\arabic{enumi})}
\tightlist
\item
  The unrestricted covariance matrix of the variables responsible for
  missingness (in this case, \(SES\) and \(SES\cdot SAT\)):
\end{enumerate}

\[
   \Sigma=
  \left[ {\begin{array}{ccc}
   \sigma^2_{SAT} & \sigma_{SAT,SES\cdot SAT} \\
    \sigma_{SES\cdot SAT, SAT} & \sigma^2_{SES \cdot SAT}  \\
  \end{array} } \right]
\]

\begin{enumerate}
\def\labelenumi{(\arabic{enumi})}
\setcounter{enumi}{1}
\tightlist
\item
  The restricted covariance matrix of all the variables (in this case,
  \(SES\), SAT, \(SES \cdot SAT\), and \(FYGPA\)):
\end{enumerate}

\[
   \widetilde{\Sigma}=
  \left[ {\begin{array}{cccc}
   \tilde{\sigma}^2_{SAT} & \tilde{\sigma}_{SAT,{SES}} & \tilde{\sigma}_{SAT,SES\cdot SAT} & \tilde{\sigma}_{SAT, FYGPA}\\
    \tilde{\sigma}_{SES,SAT} & \tilde{\sigma}^2_{SES} & \tilde{\sigma}_{SES,SES\cdot SAT} & \tilde{\sigma}_{SES, FYGPA} \\
    \tilde{\sigma}_{SES\cdot SAT,SAT} &\tilde{\sigma}_{SES\cdot SAT,SES} & \tilde{\sigma}^2_{SES\cdot SAT} & \tilde{\sigma}_{SES\cdot SAT, FYGPA} \\
     \tilde{\sigma}_{SAT, FYGPA} & \tilde{\sigma}_{SES, FYGPA} & \tilde{\sigma}_{SES\cdot SAT, FYGPA} & \tilde{\sigma}^2_{FYGPA}
  \end{array} } \right]
\]

\noindent (Note: anything with a tilde represents the restricted
estimate).

To compute \(\sigma_{SAT, SAT\cdot SES}\), we can use Equation
\ref{eq:correction}.This requires knowing the unrestricted variance of
\(SES\), which may be unavailable. However, this parameter can be
acquired using the PL correction, using incumbent data for \(SAT\). One
simply inputs the \(1\times1\) matrix of \(SAT^2\) (i.e., the variance)
as the unrestricted covariance matrix, then inputs the restricted
estimates for the covariance matrix of \(SES\)/\(SAT\).

After performing the PL correction, we now have most\footnote{The mean
  of \(SES\) may not be known, but can be estimated:
  \(\overline{SES} = b_0 + b_1\times \overline{SAT}\), where \(b_0\) and
  \(b_1\) are the regression coefficients from the model predicting
  \(SES\) from \(SAT\).} of the inputs necessary for Equation
\ref{eq:correction}. We can also compute the population covariance
between \(SES\) and \(SES \cdot SAT\) (Aiken \& West, 1991, p. 180,
Equation A.13):

\begin{equation}
cov(SES, SES\cdot SAT) = s^2_{SAT} \overline{SES} + cov(SAT, SES) \overline{SAT}
\label{eq:correction2}
\end{equation}

We also need the variance of the interaction term (Aiken \& West, 1991,
p. 179, Equation A.8):

\begin{equation}
\sigma^2_{SAT\cdot SES} = \sigma^2_{SAT}\overline{SES}^2 + \sigma^2_{SES}\overline{SAT}^2 + 2\sigma_{SES,SAT}\overline{SES}\cdot \overline{SAT} + \sigma^2_{SES}\sigma^2_{SAT} + \sigma^2_{SES,SAT}
\label{eq:correction3}
\end{equation}

(Though not necessary, we could also use Equation \ref{eq:correction2}
to estimate \(cov[SAT, SES\cdot SAT]\)). At this point, we have a
corrected variance/covariance matrix of the predictors:

\[
   \Sigma^\prime=
  \left[ {\begin{array}{ccc}
   \sigma^2_{SAT} & \sigma\prime_{SAT,{SES}} & \sigma\prime_{SAT,SES\cdot SAT} \\
    \sigma\prime_{SES,SAT} & \sigma\prime^2_{SES} & \sigma\prime_{SES,SES\cdot SAT} \\
    \sigma\prime_{SES\cdot SAT,SAT} &\sigma\prime_{SES\cdot SAT,SES} & \sigma\prime^2_{SES\cdot SAT}
  \end{array} } \right]
\] \label{eq:matrix} \noindent (Note: anything with a prime (\(\prime\))
indicates a corrected estimate).

This variance/covariance matrix can be inputted into the PL correction
(as before) to obtain a corrected covariance matrix between all
variables. I call this estimate \(r_{pl}\), for Pearson-Lawley. The Case
III correction, I call \(r_{c3}\).

To review, the PL-based correction (\(r_{pl}\)) for centered predictors
is performed as follows:

\begin{enumerate}
\item Use the PL to estimate the covariance matrix between $SES$ and $SAT$.
\item Use Equations \ref{eq:correction}-\ref{eq:correction3} to complete the third rows/columns in $\Sigma^\prime$.
\item Use the corrected $\Sigma^\prime$ to obtain the final corrected covariance matrix. 
\end{enumerate}

Recall that the corrections from Aiken and West (1991) require
symmetrical data. What is unknown is how robust \(r_{pl}\) is to
skewness. In the following section, I introduce the Monte Carlo I used
to assess \(r_{pl}\) under a variety of conditions.

\section{Method}\label{method}

To assess the performance of \(r_{pl}\), I performed a simulation by
doing the following:

\begin{enumerate}
\def\labelenumi{(\arabic{enumi})}
\tightlist
\item
  Generate \(n\) skewed \(SES\) and \(SAT\) scores, with means of 5 and
  500, respectively, and variances of one. The skewness values varied as
  shown in Table \ref{tab:mcparams}.\footnote{Many of these parameters
    were not varied because they made little difference in preliminary
    simulations. Only \(b_{sat}\), \(r\), and skew affected of bias.
    Full details of this preliminary simulation are available from the
    author.}
\item
  Center the predictor variables.
\item
  Create a product variable (\(SAT\cdot SES\)) by multiplying \(SES\)
  and \(SAT\) scores.
\item
  Generate 100 \(FYGPA\) scores, using the regression weights shown in
  Table \ref{tab:mcparams}.
\item
  Simulate selection on \(SAT\), by omitting \(SES\), \(SES\cdot SAT\),
  and \(Y\) values for those who fell below the \(p\) percentile of
  \(SAT\).
\item
  Compute the correlation between \(SES\) and \(FYGPA\) using \(r_{pl}\)
  and \(r_{c3}\).
\item
  Repeat 10,000 times.
\end{enumerate}

\begin{table}[ht]
\centering
\caption{Parameters Used for the Monte Carlo Simulation} 
\label{tab:mcparams}
\begin{tabular}{lc}
  \hline
Parameters & Values \\ 
  \hline
$b_{ses}$ & 0.242 \\ 
  $b_{sat}$ & 0, 0.099, 0.242, 0.371 \\ 
  $b_{ses\cdot sat}$ & 0.242 \\ 
  r & 0, 0.1, 0.3, 0.5 \\ 
  $\bar{ses}$ & 5 \\ 
  $\bar{sat}$ & 500 \\ 
  $n$ & 50, 100, 200, 500 \\ 
  $p_{missing}$ & 0.1, 0.3, 0.5, 0.7 \\ 
  skew & -100, -50, 0, 50, 100 \\ 
   \hline
\end{tabular}
\end{table}

Both \(r_{pl}\) and \(r_{c3}\) were averaged across conditions and
compared to the average estimates obtained from the random sample:

\[Bias = \hat{r}-r\] \noindent where \(\hat{r}\) is the estimate of
interest (either \(r_{c3}\) or \(r_{pl}\)) and \(r\) is the mean
estimate from the random sample.

\section{Results}\label{results}

Figure \ref{fig:results} shows bias as a function of skewness (\(s\)),
the correlation between \(SES\) and \(SAT\) (\(r\)), and the slope
predicting \(FYGPA\) from \(SAT\) (\(b_{ses}\), though to save space in
the plot, I have labeled it \(b\)). Each dot in the plot represents the
cell mean, averaged within the conditions labeled on the x-axis. I
labeled the various values of \(b\) only once since they repeat across
the plot and I wanted to avoid visual clutter. I also added a horizontal
line at zero to indicate where Bias = 0. The \(r_{pl}\) estimate is in
gray with closed circles, while the \(r_{c3}\) estimate is in black with
open circles.

In nearly every condition, \(r_{pl}\) outperforms \(r_{c3}\); the
\(r_{pl}\) (gray) estimates are very near the horizontal line. The only
time \(r_{c3}\) performs as good or better than \(r_{pl}\) is when
skewness is positive, and \(r\) and \(b\) are high. Otherwise,
\(r_{pl}\) always outperforms the other estimate. In addition,
\(r_{pl}\) is generally unbiased, even under heavy skew. It performs
poorest when skewness is positive, and \(r\) and \(b\) are high,
reaching approximately -0.08 (meaning the actual correlation is
underestimated by 0.08). Also, \(r_{c3}\) almost always overestimates,
while \(r_{pl}\) may underestimate or overestimate, depending on the
values of skewness, \(r\), and \(b\).

\begin{figure}[htbp]
\begin{center}
\fitfigure{bias_correction}
\caption{Average bias in estimating the correlation coefficient, under various conditions: correlation between the predictor variables ($r$), skewness ($s$), the slope between $SAT$ and $FYGPA$ ($b$), and estimator ($r_{pl}$ vs $r_{c3}$). Note that each line shows bias as a function of the values of $b$ ($b$=0.371, 0.242, 0.099, 0). These values are repeated, though only the first are labeled.}
\label{fig:results}
\end{center}
\end{figure}

\section{Discussion}\label{discussion}

Centering predictors is often recommended to enhance parameter
interpretation and reduce multicollinearity. I have shown a major
disadvantage of centering predictors: they increase bias under missing
data. Centering predictors strips \enquote{nonessential} correlation
between the interaction the predictor variables. Subsequently, the
predictors are unable to augment the missing data model and mitigate
bias.

Fortunately, there need not be a trade-off between bias and the
advantages of centering predictors. In this paper, I developed a
correction that allows researchers to center predictors when data are
missing. This correction assumes symmetrical predictors. However, the
simulation demonstrated that \(r_{pl}\) was robust to fairly extreme
skewness, and usually outperformed Case III (which assumes no
interactions exist between the predictors). Never did average bias
exceed 0.08. The Case III correction (\(r_{c3}\)), on the other hand,
performed poorly, sometimes exceeding 0.2 in bias.

Because of \(r_{pl}\)'s marginal sensitivity to skew, I recommend
caution when researchers use the correction. Univariate distributions
ought to be inspected for symmetry and transformed when appropriate. I
would not, however, recommend using \(r_{c3}\) when interactions exist.

Although \(r_{pl}\) minimizes bias when centering predictors, there is
no reason not to use it when variables are \emph{not} centered. When
variables are left uncentered, \emph{some} bias is expected (because the
data are technically MNAR). Consequently, I recommend researchers
inspect predictor/criterion relationships for potential interactions
before applying Case III. If interactions are suspected, the \(r_{pl}\)
correction will generally lead to unbiased estimates of the population
correlation.

\newpage

\section{References}\label{references}

\setlength{\parindent}{-0.5in} \setlength{\leftskip}{0.5in}

\hypertarget{refs}{}
\hypertarget{ref-aiken_multiple_1991}{}
Aiken, L. S., \& West, S. G. (1991). \emph{Multiple Regression: Testing
and Interpreting Interactions}. Newbury Park, CA: SAGE Publications,
Inc.

\hypertarget{ref-aitken_note_1935}{}
Aitken, A. C. (1935). Note on Selection from a Multivariate Normal
Population. \emph{Proceedings of the Edinburgh Mathematical Society},
\emph{4}(2), 106--110.
doi:\href{https://doi.org/10.1017/S0013091500008063}{10.1017/S0013091500008063}

\hypertarget{ref-cohen_applied_2013}{}
Cohen, J., Cohen, P., West, S. G., \& Aiken, L. S. (2013). \emph{Applied
multiple regression/correlation analysis for the behavioral sciences}.
New York, NY: Routledge.

\hypertarget{ref-collins_comparison_2001}{}
Collins, L. M., Schafer, J. L., \& Kam, C. M. (2001). A comparison of
inclusive and restrictive strategies in modern missing data procedures.
\emph{Psychological Methods}, \emph{6}(4), 330--351.

\hypertarget{ref-dalal_common_2012}{}
Dalal, D. K., \& Zickar, M. J. (2012). Some Common Myths About Centering
Predictor Variables in Moderated Multiple Regression and Polynomial
Regression. \emph{Organizational Research Methods}, \emph{15}(3),
339--362.
doi:\href{https://doi.org/10.1177/1094428111430540}{10.1177/1094428111430540}

\hypertarget{ref-echambadi_mean-centering_2007}{}
Echambadi, R., \& Hess, J. D. (2007). Mean-Centering Does Not Alleviate
Collinearity Problems in Moderated Multiple Regression Models.
\emph{Marketing Science}, \emph{26}(3), 438--445.
doi:\href{https://doi.org/10.1287/mksc.1060.0263}{10.1287/mksc.1060.0263}

\hypertarget{ref-kromrey_mean_1998}{}
Kromrey, J. D., \& Foster-Johnson, L. (1998). Mean Centering in
Moderated Multiple Regression: Much Ado about Nothing. \emph{Educational
and Psychological Measurement}, \emph{58}(1), 42--67.
doi:\href{https://doi.org/10.1177/0013164498058001005}{10.1177/0013164498058001005}

\hypertarget{ref-lawley_note_1944}{}
Lawley, D. N. (1944). A Note on Karl Pearson's Selection Formulae.
\emph{Proceedings of the Royal Society of Edinburgh. Section A.
Mathematical and Physical Sciences}, \emph{62}(01), 28--30.
doi:\href{https://doi.org/https://doi.org/10.1017/S0080454100006385}{https://doi.org/10.1017/S0080454100006385}

\hypertarget{ref-little_statistical_2014}{}
Little, R. J. A., \& Rubin, D. B. (2014). \emph{Statistical Analysis
with Missing Data}. Hoboken, NJ: John Wiley \& Sons.

\hypertarget{ref-rubin_inference_1976}{}
Rubin, D. B. (1976). Inference and missing data. \emph{Biometrika},
\emph{63}(3), 581--592.
doi:\href{https://doi.org/10.1093/biomet/63.3.581}{10.1093/biomet/63.3.581}






\end{document}
