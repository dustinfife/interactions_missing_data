\documentclass[english,man]{apa6}

\usepackage{amssymb,amsmath}
\usepackage{ifxetex,ifluatex}
\usepackage{fixltx2e} % provides \textsubscript
\ifnum 0\ifxetex 1\fi\ifluatex 1\fi=0 % if pdftex
  \usepackage[T1]{fontenc}
  \usepackage[utf8]{inputenc}
\else % if luatex or xelatex
  \ifxetex
    \usepackage{mathspec}
    \usepackage{xltxtra,xunicode}
  \else
    \usepackage{fontspec}
  \fi
  \defaultfontfeatures{Mapping=tex-text,Scale=MatchLowercase}
  \newcommand{\euro}{€}
\fi
% use upquote if available, for straight quotes in verbatim environments
\IfFileExists{upquote.sty}{\usepackage{upquote}}{}
% use microtype if available
\IfFileExists{microtype.sty}{\usepackage{microtype}}{}

% Table formatting
\usepackage{longtable, booktabs}
\usepackage{lscape}
% \usepackage[counterclockwise]{rotating}   % Landscape page setup for large tables
\usepackage{multirow}		% Table styling
\usepackage{tabularx}		% Control Column width
\usepackage[flushleft]{threeparttable}	% Allows for three part tables with a specified notes section
\usepackage{threeparttablex}            % Lets threeparttable work with longtable

% Create new environments so endfloat can handle them
% \newenvironment{ltable}
%   {\begin{landscape}\begin{center}\begin{threeparttable}}
%   {\end{threeparttable}\end{center}\end{landscape}}

\newenvironment{lltable}
  {\begin{landscape}\begin{center}\begin{ThreePartTable}}
  {\end{ThreePartTable}\end{center}\end{landscape}}

  \usepackage{ifthen} % Only add declarations when endfloat package is loaded
  \ifthenelse{\equal{\string man}{\string man}}{%
   \DeclareDelayedFloatFlavor{ThreePartTable}{table} % Make endfloat play with longtable
   % \DeclareDelayedFloatFlavor{ltable}{table} % Make endfloat play with lscape
   \DeclareDelayedFloatFlavor{lltable}{table} % Make endfloat play with lscape & longtable
  }{}%



% The following enables adjusting longtable caption width to table width
% Solution found at http://golatex.de/longtable-mit-caption-so-breit-wie-die-tabelle-t15767.html
\makeatletter
\newcommand\LastLTentrywidth{1em}
\newlength\longtablewidth
\setlength{\longtablewidth}{1in}
\newcommand\getlongtablewidth{%
 \begingroup
  \ifcsname LT@\roman{LT@tables}\endcsname
  \global\longtablewidth=0pt
  \renewcommand\LT@entry[2]{\global\advance\longtablewidth by ##2\relax\gdef\LastLTentrywidth{##2}}%
  \@nameuse{LT@\roman{LT@tables}}%
  \fi
\endgroup}


\ifxetex
  \usepackage[setpagesize=false, % page size defined by xetex
              unicode=false, % unicode breaks when used with xetex
              xetex]{hyperref}
\else
  \usepackage[unicode=true]{hyperref}
\fi
\hypersetup{breaklinks=true,
            pdfauthor={},
            pdftitle={When Interactions Bias Corrections: A Missing Data Correction for Centered Predictors},
            colorlinks=true,
            citecolor=blue,
            urlcolor=blue,
            linkcolor=black,
            pdfborder={0 0 0}}
\urlstyle{same}  % don't use monospace font for urls

\setlength{\parindent}{0pt}
%\setlength{\parskip}{0pt plus 0pt minus 0pt}

\setlength{\emergencystretch}{3em}  % prevent overfull lines

\ifxetex
  \usepackage{polyglossia}
  \setmainlanguage{}
\else
  \usepackage[english]{babel}
\fi

% Manuscript styling
\captionsetup{font=singlespacing,justification=justified}
\usepackage{csquotes}
\usepackage{upgreek}

 % Line numbering
  \usepackage{lineno}
  \linenumbers


\usepackage{tikz} % Variable definition to generate author note

% fix for \tightlist problem in pandoc 1.14
\providecommand{\tightlist}{%
  \setlength{\itemsep}{0pt}\setlength{\parskip}{0pt}}

% Essential manuscript parts
  \title{When Interactions Bias Corrections: A Missing Data Correction for
Centered Predictors}

  \shorttitle{Interactions bias correlations}


  \author{Dustin Fife\textsuperscript{1}}

  \def\affdep{{""}}%
  \def\affcity{{""}}%

  \affiliation{
    \vspace{0.5cm}
          \textsuperscript{1} Rowan University  }

 % If no author_note is defined give only author information if available
      \newcounter{author}
                              \authornote{
          Correspondence concerning this article should be addressed to Dustin Fife
          , 201 Mullica Hill Road Glassboro, NJ 08028 .
           E-mail: \href{mailto:fife.dustin@gmail.com}{\nolinkurl{fife.dustin@gmail.com}} 
          }
                    

  \abstract{It is commonly advised to center predictors in multiple regression,
especially in the presence of interactions or polynomial terms (J.
Cohen, Cohen, West, \& Aiken, 2013). This will enhance the
interpretation of regression parameters, and (arguably; Dalal \& Zickar,
2012; Echambadi \& Hess, 2007; Kromrey \& Foster-Johnson, 1998) will
reduce multicollinearity. However, in this paper, I demonstrate that in
some missing data situations, centering predictors biases parameter
estimates and decreases precision. I also develop a
Pearson-Lawyley-based (Aitken, 1935; Lawley, 1944) missing data
correction (called \(r_{pl}\)) that does not require uncentered
predictors, then evaluate the performance of this correction via Monte
Carlo Simulation.}
  \keywords{missing data, selection, range restriction, interactions \\

    \indent Word count: X
  }




  \usepackage{colortbl}

\usepackage{amsthm}
\newtheorem{theorem}{Theorem}
\newtheorem{lemma}{Lemma}
\theoremstyle{definition}
\newtheorem{definition}{Definition}
\newtheorem{corollary}{Corollary}
\newtheorem{proposition}{Proposition}
\theoremstyle{definition}
\newtheorem{example}{Example}
\theoremstyle{remark}
\newtheorem*{remark}{Remark}
\begin{document}

\maketitle

\setcounter{secnumdepth}{0}



It is commonly believed (and stated) that when performing a multiple
regression, researchers ought to center predictor variables (which
consists of subtracting the mean of the predictor variable{[}s{]} from
every score; J. Cohen et al., 2013). Doing so often improves the
interpretation of parameter estimates, especially when zero has no
meaningful interpretation (e.g., IQ). It has also been suggested that
centering predictors reduces multicollinearity, thereby increasing
precision of parameter estimates (J. Cohen et al., 2013). Although this
last advantage has been hotly debated (Dalal \& Zickar, 2012; Echambadi
\& Hess, 2007; Kromrey \& Foster-Johnson, 1998), none (that I know of)
have recommended against centering predictors (except when the metric of
the predictor variable has a meaningful zero point). However, there is
one situation where centering predictors will not only \emph{decrease}
precision, but it will also bias parameter estimates.

In this paper, I show how centering predictors can lead to substantial
bias when data are missing (such as when subjects drop out of a study or
some sort of selection procedure is operating). In order to do so, I
will first briefly review the literature on centering predictors and
show the mathematical advantages of doing so. After which, I will review
the missing data literature and show how centering predictors may change
a \enquote{Missing At Random} situation into on that is \enquote{Missing
\emph{Not} at Random,} which I will then highlight via a simulated
example. Finally, I will introduce a correction that allows researchers
to center predictors without bias, and assess its performance via Monte
Carlo Simulation.

\subsection{Regression and Centering}\label{regression-and-centering}

In a multiple regression context, interactions often exist between two
or more predictor variables. Suppose, for example, an academic
institution is interested in assessing the impact of socioeconomic
status (\(SES\)) on \(FYGPA\), and wishes to correct for missing data
(in this case, let us assume the university selected based on \(SAT\)
scores). Further suppose that these two predictor variables interact,
such that for those with high \(SAT\) scores, \(SES\) is more predictive
of \(FYGPA\) scores than for those with lower \(SAT\) scores.
Mathematically,

\[ FYGPA = b_0 + b_1SES + b_2SAT + b_3SES\cdot SAT \] where \(b_3\) will
be some positive value (indicating that as \(SAT\) gets higher, \(SES\)
will become more predictive of FYGPA). In this situation, the researcher
might be inclined to center both \(SES\) and \(SAT\) scores. Doing so
supposedly has two advantages. First, the coefficients for the
transformed variables will have a more sensible interpretation (J. Cohen
et al., 2013). The original zero points for the predictors are
meaningless, which means that the intercept parameter is of little
interest (in this case, the predicted \(FYGPA\) for someone who has an
SAT/\(SES\) of zero). Centering these predictors now yields a meaningful
interpretation (the predicted \(FYGPA\) for someone who has an average
\(SAT\) and average \(SES\) score).

This interpretive advantage is heightened when interactions are present
in the model. Recall that when an interaction is present, the
relationship between, say \(SES\) and \(FYGPA\) is non-linear; the slope
between \(SES\) and \(FYGPA\) changes depending on the value of \(SAT\).
When centered, the \(b_1\) parameter, for example, is the \emph{average}
slope of \(FYGPA\) on \(SES\) across all values of \(SAT\).

The second purported advantage of centering is that is removes
\enquote{nonessential} multicollinearity (Aiken \& West, 1991; J. Cohen
et al., 2013). Mathematically, the covariance between the interaction
variable (\(SES\cdot SAT\)) and either predictor is a function of the
arithmetic means of \(SAT\) and \(SES\) (Aiken \& West, 1991, p. 180,
Equation A.13):

\begin{equation}
cov(SES, SES\cdot SAT) = s^2_{SES} \overline{SAT} + cov(SAT, SES) \overline{SES}
\label{eq:correction}
\end{equation}

\noindent (Note that the above equation only applies when each predictor
is completely symmetrical). When both predictors are centered, the means
become zero and the covariance between the two vanishes. This is what we
call \enquote{nonessential multicollinearity,} or the collinearity that
is attributable to the means of the predictors.

When the predictors are \emph{not} symmetrical, some relationship
between the two will remain. What remains is what is called
\enquote{essential} multicollinearity.

Some (e.g., J. Cohen et al., 2013) argue that removing essential
multicollinearity will increase the precision of parameter estimates.
The rationale is simple, as multicollinearity tends to inflate standard
errors. However, others (Dalal \& Zickar, 2012; Echambadi \& Hess, 2007;
Kromrey \& Foster-Johnson, 1998) have demonstrated mathematically that
precision is unaffected by centering.

Regardless of how it does (or does not) affect precision, centering is
often considered wise practice, if at least for its interpretative
advantages. However, when data are missing (such as due to selection or
attrition), centering may inflate bias and standard errors.

\section{Missing Data}\label{missing-data}

To understand how centering may exacerbate bias, let us review the
missing data nomenclature. Rubin (1976) developed a framework under
which to view missing data. He considered three missing data situations:
Missing Completely at Random (MCAR), Missing at Random (MAR), and
Missing Not At Random (MNAR; Little \& Rubin, 2014; Rubin, 1976).

\subsection{MCAR}\label{mcar}

When data are MCAR, the probability of missingness is unrelated to
either the observable or the unobservable data. Put differently, those
values missing can be considered a random sample of all the available
data (Graham, 2012). In this situation, nearly all methods of handling
missing data (e.g., listwise deletion, mean imputation, maximum
likelihood) will yield unbiased parameter estimates (Enders, 2010;
Graham, 2012).

As an example, suppose some of the students have missing \(SES\) scores
because of a computer outage that selectively wiped some students' data.
Because the probability of a computer outage is unlikely to be related
to \(FYGPA\), this is a MCAR situation.

\subsection{MAR}\label{mar}

When data are MAR, the probability of missingness is correlated with the
observable data, but \emph{not} the unobserved data. For example,
suppose the university selected students into the university based on
their \(SAT\) scores. Naturally, those not selected will be missing
\(FYGPA\) scores. However, because we have measured and recorded the
cause of missingness (SAT scores), it is possible to obtain unbiased
estimates of model parameters, provided that the appropriate method of
analysis is used (e.g., maximum likelihood methods or multiple
imputation).

\subsection{MNAR}\label{mnar}

Finally, when data are MNAR, the probability of missingness is
correlated with \emph{both} the observable data and unobservable data.
For example, suppose at the end of the first academic year, not only are
\(FYGPA\) scores missing for those who were not selected into the
university, but some are missing because they dropped out of the
university due to lack of motivation. In this instance, motivation is
the cause of missingness, but because the researcher did not measure
motivation, the cause of missingness is unobservable.

When data are MNAR, it is difficult to obtain unbiased estimates without
making quite restrictive assumptions (Enders, 2010; Heckman, 1979).

\subsection{Interactions and Missing
Data}\label{interactions-and-missing-data}

In concurrent validity designs, when interactions exist in a regression
model, the data are technically MNAR. To understand why, consider our
previous example. Again, suppose students were selected based on \(SAT\)
scores. Now let us further suppose the researcher is interested in
assessing the correlation between socioeconomic status (\(SES\)) and
\(FYGPA\) on the current cohort of applicants. However, they want to
know the correlation in the unselected population, but unfortunately
only have applicant data for \(SES\). Assuming \(SES\) itself is not a
cause of attrition (or selection), missingness was actually cased by two
variables:

\begin{enumerate}
\def\labelenumi{(\arabic{enumi})}
\tightlist
\item
  \(SAT\) scores. Since these were recorded before selection, these data
  are technically observable and missingness due to this is MAR.
\item
  \(SAT\cdot SES\) scores. This product term is correlated with the
  probability of missingness in such a way that is independent of
  \(SAT\) and \(SES\) alone (since an interaction is present). Some of
  these product scores are missing (because they were not selected into
  the university), rending them unobservable. Because we have an
  independent correlate of missingness (the product) that cannot be
  observed (because scores of students not selected into the university
  are missing), the data are MNAR.
\end{enumerate}

Notice that the data are MNAR, regardless of whether \(SES\) itself is a
cause of missingness (again, because the product variable is missing for
certain applicants). Had \(SES\) been measured before selection on
\(SAT\) occurred (i.e., in a predictive validity design), the data would
be MAR.

In most situations, the fact that the data are MNAR is not altogether
problematic. Recall that one need not actually model the cause of
missingness to render a situation MAR. Rather, one simply needs to model
a correlate of the cause of missingness (Collins, Schafer, \& Kam,
2001). With uncentered variables, the correlation between each of the
predictors and their product is quite high and thus, even though the
product term is a cause of missingness, we can actually control for it
using the applicant \(SAT\) scores. When we center the variables,
however, that correlation vanishes and the MNAR-ness of the data is
exacerbated.

Naturally, a resourceful researcher might decide to include the
interaction term as a predictor in a regression model, assuming that by
including the cause of missingness they will render the data MNAR. Alas,
this is not so because the product scores (\(SES\cdot SAT\)) for those
who were not selected into the university are still missing.
Consequently, without modification to the algorithms, there is no way to
obtain an unbiased estimate using current missing data techniques.

\section{Demonstration}\label{demonstration}

To illustrate this problem (centering predictors exacerbates bias when
interactions are present), I performed a simulation by doing the
following\footnote{Complete access to the code that generated the data
  is freely available from the author:}:

\begin{enumerate}
\def\labelenumi{(\arabic{enumi})}
\tightlist
\item
  Generate 100 fictitious \(SES\) and \(SAT\) scores. These scores were
  generated from a random normal distribution. The means were set to 10
  and 3, respectively, their standard deviations to 3 and .2, and their
  covariance to 90 (which is equivalent to a 0.3 correlation).
\item
  Standardize the predictor variables. On half of the iterations (see
  step 7), the predictor variables were centered around their mean.
\item
  Create a product variable (\(SAT\cdot SES\)) by multiplying \(SES\)
  and \(SAT\) scores.
\item
  Generate 100 \(FYGPA\) scores, using the following equation:
  \[ FYGPA = b_0 + b_1SES + b_2SAT + b_3SES\cdot SAT + e\] The values
  for \(b_0-b_3\) were chosen such that \(FYGPA\) had an expected value
  of 3.0 and a standard deviation of 0.4. The standardized slopes
  (\(b_1-b_3\)) were set to 0.3. (Note that the value of unstandardized
  values of \(b_0-b_2\) changed depending on whether the current
  iteration was standardized.)
\item
  Simulate selection on \(SAT\). To do this, I set \(SES\),
  \(SES\cdot SAT\), and \(Y\) to missing for those individuals who had
  \(SAT\) scores below the mean (approximately 0 or 500, depending on
  whether the current iteration was standardized). For comparison, I
  also created a separate dataset which was simply a random sample of
  half the scores.
\item
  Compute the corrected and uncorrected correlation between \(SES\) and
  \(FYGPA\). To correct, I used both the expectation maximization (EM)
  algorithm (via the em.norm function in the norm package in R; Schafer,
  Novo, \& Fox, 2010), as well as the Case III correction (via the
  caseIII function in the selection package in R; Fife, 2016). Note that
  the standard Case III correction ignores the fact that there is an
  interaction present. For comparison, I also computed the simple
  correlation in the random sample.
\item
  Repeat 10,000 times. To estimate bias and assess standard errors,
  these steps were repeated 10,000 times.
\end{enumerate}

The results of this simulation are presented in Figure \ref{fig:demo}.
The shaded boxes represent the distribution of estimates from the
uncentered conditions while the open boxes represent those from the
centered ones. Notice that, even when the variables are not centered,
Case III and the EM are biased, though not by much (an average of 0.01
for both estimates). Again, the reason they are only slightly biased is
because \(SAT\) is highly correlated with \(SAT\cdot SES\). When
centered, however, bias is much worse (an average of 0.18 for both). In
addition, standard errors are slightly larger when centered (0.12 for
Case III/EM in the centered condition and 0.11 in the uncentered
condition).

\begin{figure}[htbp]
\begin{center}
\scalebox{0.75}{\fitfigure{../plots/demonstration}}
\caption{Boxplots showing the distribution of two correction procedures (Case III and the EM algorithm) relative to a random sample, across 10,000 iterations. The shaded boxes represent the distribution of estimates from the uncentered predictors, while the un-shaded boxes are from the centered predictors. }
\label{fig:demo}
\end{center}
\end{figure}

These results demonstrate a clear advantage to \emph{not} centering
variables when an interaction is present (at least when missing data are
involved).

\section{Potential Solutions}\label{potential-solutions}

The obvious solution to the bias problem is to simply not center
variables. However, this may not be ideal if a researcher is keen on the
interpretive benefits of centering. Consequently, I offer a correction.

Recall that the Pearson-Lawley correction (Aitken, 1935; Lawley, 1944)
provides a way to correct estimates for missing data that occurs on one
or more variables. It is a multivariate extension of the traditional
Case III correction and requires two inputs:

\begin{enumerate}
\def\labelenumi{(\arabic{enumi})}
\tightlist
\item
  The unrestricted (unbiased) variance/covariance matrix of the
  variables responsible for missingness. In this case, that would be
  \(SES\) and \(SES\cdot SAT\).
\item
  The restricted (biased) variance/covariance matrix of all the
  variables in the model. In this case, that would be \(SES\), SAT,
  \(SES \cdot SAT\), and \(FYGPA\).
\end{enumerate}

In matrix form, we need the following population variance/covariance
matrix

\[
   \Sigma=
  \left[ {\begin{array}{ccc}
   \sigma^2_{SAT} & \sigma_{SAT,SES\cdot SAT} \\
    \sigma_{SES\cdot SAT, SAT} & \sigma^2_{SES \cdot SAT}  \\
  \end{array} } \right]
\]

And the following restricted variance/covariance matrix

\[
   \widetilde{\Sigma}=
  \left[ {\begin{array}{cccc}
   \tilde{\sigma}^2_{SAT} & \tilde{\sigma}_{SAT,{SES}} & \tilde{\sigma}_{SAT,SES\cdot SAT} & \tilde{\sigma}_{SAT, FYGPA}\\
    \tilde{\sigma}_{SES,SAT} & \tilde{\sigma}^2_{SES} & \tilde{\sigma}_{SES,SES\cdot SAT} & \tilde{\sigma}_{SES, FYGPA} \\
    \tilde{\sigma}_{SES\cdot SAT,SAT} &\tilde{\sigma}_{SES\cdot SAT,SES} & \tilde{\sigma}^2_{SES\cdot SAT} & \tilde{\sigma}_{SES\cdot SAT, FYGPA} \\
     \tilde{\sigma}_{SAT, FYGPA} & \tilde{\sigma}_{SES, FYGPA} & \tilde{\sigma}_{SES\cdot SAT, FYGPA} & \tilde{\sigma}^2_{FYGPA}
  \end{array} } \right]
\]

\noindent (Note: anything with a tilde represents the restricted
estimate).

To compute the covariance between \(SAT\) and \(SAT\cdot SES\), we can
use Equation \ref{eq:correction}. Unfortunately, this requires knowing
\(SES^2\), or the unrestricted (unbiased) variance of \(SES\). Because
these data were collected on incumbents, we don't have access to that
information. However, this parameter can be acquired using the PL
correction, assuming we have access to the incumbent data for \(SAT\).
(If that information is unavailable, we direct the reader to Fife,
Hunter, \& Mendoza, 2016, who offer corrections for situations where
incumbent data is unavailable). One simply inputs the \(1\times1\)
matrix of \(SAT^2\) (i.e., the variance) as the unrestricted
variance/covariance matrix, then subsequently, inputs the restricted
estimates for the \(SES\)/\(SAT\) variances, as well as their
covariance.

After performing the PL correction, we now have most\footnote{The mean
  of the incidentally restricted variable (\(SES\) in this case) may not
  be known. However, one can estimate this using the following equation:
  \(\overline{SES} = b_0 + b_1\times \overline{SAT}\), where \(b_0\) and
  \(b_1\) are the regression coefficients from the model predicting
  \(SES\) from \(SAT\).} of the inputs necessary for Equation
\ref{eq:correction}. While we are at it, we might as well compute the
population covariance between \(SES\) and \(SES \cdot SAT\) (Aiken \&
West, 1991, p. 180, Equation A.13):

\begin{equation}
cov(SES, SES\cdot SAT) = s^2_{SAT} \overline{SES} + cov(SAT, SES) \overline{SAT}
\label{eq:correction2}
\end{equation}

And, of course, we need the variance of the interaction term (Aiken \&
West, 1991, p. 179, Equation A.8):

\begin{equation}
\sigma^2_{SAT\cdot SES} = \sigma^2_{SAT}\overline{SES}^2 + \sigma^2_{SES}\overline{SAT}^2 + 2\sigma_{SES,SAT}\overline{SES}\cdot \overline{SAT} + \sigma^2_{SES}\sigma^2_{SAT} + \sigma^2_{SES,SAT}
\label{eq:correction3}
\end{equation}

At this point, we have a corrected variance/covariance matrix of the
predictors:

\[
   \Sigma^\prime=
  \left[ {\begin{array}{ccc}
   \sigma^2_{SAT} & \sigma\prime_{SAT,{SES}} & \sigma\prime_{SAT,SES\cdot SAT} \\
    \sigma\prime_{SES,SAT} & \sigma\prime^2_{SES} & \sigma\prime_{SES,SES\cdot SAT} \\
    \sigma\prime_{SES\cdot SAT,SAT} &\sigma\prime_{SES\cdot SAT,SES} & \sigma\prime^2_{SES\cdot SAT}
  \end{array} } \right]
\] \label{eq:matrix} \noindent (Note: anything with a prime (\(\prime\))
indicates the estimate has been corrected).

This variance/covariance matrix can then be inputted into the PL
equation (as before) to obtain a doubly corrected variance/covariance
matrix between the predictors and the outcome. For simplicity, we will
call this estimate \(r_{pl}\), for Pearson-Lawley. The standard
correction (using Case III and ignoring the interaction term), we will
call \(r_{c3}\).

To review, the PL-based correction (\(r_{pl}\)) for centered predictors
is performed as follows:

\begin{enumerate}
\item Use the PL to estimate the variance/covariance matrix between $SES$ and SAT
\item Use Equations \ref{eq:correction}-\ref{eq:correction3} to complete the third rows/columns in $\Sigma^\prime$.
\item Use the corrected $\Sigma^\prime$ to obtain the final corrected variance/covariance matrix. 
\end{enumerate}

Recall that the corrections from Aiken and West (1991) require that the
data are symmetrical. What is unknown is how robust this correction is
in the presence of skewness. It is also unknown how this correction
fares in terms of standard errors. In the following section, I introduce
the Monte Carlo Simulation I used to assess the performance of the
correction under a variety of conditions.

\section{Method}\label{method}

The Monte Carlo simulation was nearly identical to the simulation in the
demonstration, with the exception of the parameters varied. The
parameters varied are shown in Table \ref{tab:mcparams}.\footnote{Many
  of these parameters were not varied because they made little to no
  difference in preliminary simulations. These preliminary simulations
  randomly varied every parameter using a random uniform distribution.
  Subsequently, a Random Forest (RF; Breiman, 2001) model was used to
  determine which parameters were predictive of bias. The benefit of RF
  is that it natively detects interactions, which is clearly important
  in this situation. After performing the RF, only \(b_{sat}\), \(r\),
  and skew were predictive of bias. (Note that we only predicted bias
  for the \(r_{pl}\) estimate. Had we also predicted bias for the
  \(r_{c3}\) estimate, other variables may have also been predictive).
  We also varied \(p_{missing}\) and \(n\) since they will affect
  standard errors. Full details of this preliminary simulation are
  available from the author.} In short, I did the following:

\begin{table}[ht]
\centering
\caption{Parameters Used for the Monte Carlo Simulation} 
\label{tab:mcparams}
\begin{tabular}{lc}
  \hline
Parameters & Values \\ 
  \hline
$b_{ses}$ & 0.242 \\ 
  $b_{sat}$ & 0, 0.099, 0.242, 0.371 \\ 
  $b_{ses\cdot sat}$ & 0.242 \\ 
  r & 0, 0.1, 0.3, 0.5 \\ 
  $\bar{ses}$ & 5 \\ 
  $\bar{sat}$ & 500 \\ 
  $n$ & 50, 100, 200, 500 \\ 
  $p_{missing}$ & 0.1, 0.3, 0.5, 0.7 \\ 
  skew & -100, -50, 0, 50, 100 \\ 
   \hline
\end{tabular}
\end{table}

\begin{enumerate}
\def\labelenumi{(\arabic{enumi})}
\tightlist
\item
  Generate \(n\) skewed \(SES\) and \(SAT\) scores, with means of 5 and
  500, respectively, and variances of one. The skewness values varied as
  shown in Table \ref{tab:mcparams}.
\item
  Center the predictor variables.
\item
  Create a product variable (\(SAT\cdot SES\)) by multiplying \(SES\)
  and \(SAT\) scores.
\item
  Generate 100 \(FYGPA\) scores, using the regression weights shown in
  Table \ref{tab:mcparams}.
\item
  Simulate selection on \(SAT\), by omitting \(SES\), \(SES\cdot SAT\),
  and \(Y\) values for those who fell below the \(p\) percentile of
  \(SAT\).
\item
  Compute the correlation between \(SES\) and \(FYGPA\) using \(r_{pl}\)
  and \(r_{c3}\).
\item
  Repeat 10,000 times.
\end{enumerate}

There is one other detail worth mentioning. The population value of the
correlation is less tractable when the data are skewed. Since skewness
tends to attenuate correlation coefficients, I instead compared the
average \(r_{pl}\) (i.e., averaged across the conditions listed in Table
\ref{tab:mcparams}) and \(r_{c3}\) values to the averaged random sample
values. In the results that follow, bias values are reported relative to
the random sample. That is,

\[Bias = \hat{r}-r\] \noindent where \(\hat{r}\) is the estimate of
interest (either \(r_{c3}\) or \(r_{pl}\)) and \(r\) is the mean
estimate from the random sample.

\section{Results}\label{results}

Figure \ref{fig:results} shows how bias changed as a function of
skewness (\(s\)), the correlation between \(SES\) and \(SAT\) (\(r\)),
and the slope predicting \(FYGPA\) from \(SAT\) (\(b_{ses}\), though to
save space in the plot, I have labeled it \(b\)). Each dot in the plot
represents the mean, collapsed across the conditions labeled on the
x-axis. Note that I have only labeled the various values of \(b\) only
once since they repeat across the plot and I wanted to avoid visual
clutter. I have also added a horizontal line at zero to indicate where
Bias = 0. The \(r_{pl}\) estimate is in gray with closed circles, while
the \(r_{c3}\) estimate is in black with open circles.

The first thing to notice is that, in nearly every condition, \(r_{pl}\)
outperforms \(r_{c3}\); across nearly all conditions, the \(r_{pl}\)
(gray) estimates are very near the horizontal line. The only time
\(r_{c3}\) performs as good or better than \(r_{pl}\) is when skewness
is positive, and \(r\) and \(b\) are high. Otherwise, \(r_{pl}\) always
outperforms the other estimate. In addition, \(r_{pl}\) is generally
unbiased, even under fairly heavy skew. It performs poorest when
skewness is positive, and \(r\) and \(b\) are high, reaching values of
approximately -0.08 (meaning the actual correlation is underestimated by
0.08). It is also worth noting that \(r_{c3}\) almost always
overestimates, while \(r_{pl}\) may underestimate or overestimate,
depending on the values of skewness, \(r\), and \(b\).

\begin{figure}[htbp]
\begin{center}
\fitfigure{bias_correction}
\caption{Average bias in estimating the correlation coefficient, under various conditions: correlation between the predictor variables ($r$), skewness ($s$), the slope between $SAT$ and $FYGPA$ ($b$), and estimator ($r_{pl}$ vs $r_{c3}$). Note that each line shows bias as a function of the values of $b$ ($b$=0.371, 0.242, 0.099, 0). These values are repeated, though only the first are labeled.}
\label{fig:results}
\end{center}
\end{figure}

Figure \ref{fig:se} shows the empirical standard errors from the same
simulation. Here, standard errors are plotted against \(n\), proportion
missing (\(p\)), and skewness (\(s\)). As before, the light-colored line
(with solid circles) is the \(r_{pl}\) estimate and the dark line (with
open circles) is the \(r_{c3}\) estimate.

Not surprisingly, standard errors increase as \(n\) decreases and as
\(p\) increases. In addition, skewness also influences standard errors;
as data become more negatively skewed, standard errors increase, at
least when more than 50\% of data are missing. Finally, \(r_{pl}\) has
larger standard errors than \(r_{c3}\), at least when a large proportion
of data are missing (\textgreater{}50\%). This advantage is much smaller
for lower values of \(p\).

\begin{figure}[htbp]
\begin{center}
\fitfigure{se_correction}
\caption{Standard errors in estimating the correlation coefficient, under various conditions: sample size ($n$), skewness ($s$), the proportion of missing data ($p$), and estimator ($r_{pl}$ vs $r_{c3}$). Note that each line shows bias as a function of the values of $s$ ($s$=100, 50, 0, -50, -100). These values are repeated, though only the first are labeled.}
\label{fig:se}
\end{center}
\end{figure}

\section{Discussion}\label{discussion}

Centering predictors in multiple regression is often recommended as a
method of enhancing the interpretation of parameters and reducing
multicollinearity. In this paper, I have shown that a major disadvantage
of centering predictors is that it may increase bias and decrease
precision when data are missing. The reason is because centering
predictors strips \enquote{nonessential} correlation between the
interaction variable and the outcome. The net result of this is that
other variables in the model are unable to augment the missing data
model and mitigate bias.

Fortunately, there need not be a trade-off between bias and the
advantages of centering predictors. In this paper, I have developed a
correction, which allows researchers to center predictors even when data
are missing. Unfortunately, this correction relies on the assumption of
skewness. However, the Monte Carlo simulation demonstrated that this
correction (\(r_{pl}\)) was generally robust to fairly extreme skewness,
and usually outperformed the standard Case III correction (which assumes
no interactions exist between the predictor variables). Never did
average bias values exceed 0.08. The Case III correction (\(r_{c3}\)),
on the other hand, performed quite poorly; sometimes it exceeded 0.2 in
bias, though it did tend to have smaller standard errors than the PL
correction (at least when the proportion missing was more than 50\%).

Because of \(r_{pl}\)'s marginal sensitivity to skew, I recommend
caution when researchers attempt to use the correction. Univariate
distributions ought to be inspected for symmetry and, when not
symmetric, transformations may be applied. I would not, however,
recommend using the standard Case III correction. As this simulation
shows, Case III tends to over-correct when interactions exist.

Although \(r_{pl}\) is intended to minimize bias when centering
predictors, there is no reason not to use it when variables are
\emph{not} centered. As shown in Figure \ref{fig:demo}, even if
variables are left uncentered, some bias is expected. In this
demonstration, average bias values only reached 0.01. However, there may
be a different set of values (e.g., correlations between the predictors,
sample sizes, means, variances) that might yield substantial bias.
Consequently, I recommend applied researchers always inspect
predictor/criterion relationships for potential interactions before
applying Case III. If interactions are suspected, the \(r_{pl}\)
correction will generally lead to unbiased estimates of the population
correlation.

\newpage

\section{References}\label{references}

\setlength{\parindent}{-0.5in} \setlength{\leftskip}{0.5in}

\hypertarget{refs}{}
\hypertarget{ref-aiken_multiple_1991}{}
Aiken, L. S., \& West, S. G. (1991). \emph{Multiple Regression: Testing
and Interpreting Interactions}. Newbury Park, CA: SAGE Publications,
Inc.

\hypertarget{ref-aitken_note_1935}{}
Aitken, A. C. (1935). Note on Selection from a Multivariate Normal
Population. \emph{Proceedings of the Edinburgh Mathematical Society},
\emph{4}(2), 106--110.
doi:\href{https://doi.org/10.1017/S0013091500008063}{10.1017/S0013091500008063}

\hypertarget{ref-breiman_random_2001}{}
Breiman, L. (2001). Random Forests. \emph{Machine Learning},
\emph{45}(1), 5--32.
doi:\href{https://doi.org/10.1023/A:1010933404324}{10.1023/A:1010933404324}

\hypertarget{ref-cohen_applied_2013}{}
Cohen, J., Cohen, P., West, S. G., \& Aiken, L. S. (2013). \emph{Applied
multiple regression/correlation analysis for the behavioral sciences}.
New York, NY: Routledge.

\hypertarget{ref-collins_comparison_2001}{}
Collins, L. M., Schafer, J. L., \& Kam, C. M. (2001). A comparison of
inclusive and restrictive strategies in modern missing data procedures.
\emph{Psychological Methods}, \emph{6}(4), 330--351.

\hypertarget{ref-dalal_common_2012}{}
Dalal, D. K., \& Zickar, M. J. (2012). Some Common Myths About Centering
Predictor Variables in Moderated Multiple Regression and Polynomial
Regression. \emph{Organizational Research Methods}, \emph{15}(3),
339--362.
doi:\href{https://doi.org/10.1177/1094428111430540}{10.1177/1094428111430540}

\hypertarget{ref-echambadi_mean-centering_2007}{}
Echambadi, R., \& Hess, J. D. (2007). Mean-Centering Does Not Alleviate
Collinearity Problems in Moderated Multiple Regression Models.
\emph{Marketing Science}, \emph{26}(3), 438--445.
doi:\href{https://doi.org/10.1287/mksc.1060.0263}{10.1287/mksc.1060.0263}

\hypertarget{ref-enders_applied_2010}{}
Enders, C. K. (2010). \emph{Applied missing data analysis.} New York,
NY: Guilford Press.

\hypertarget{ref-fife_selection:_2016}{}
Fife, D. (2016). Selection: Estimating unattenuated correlations on
range restricted correlations. Retrieved from
\url{https://CRAN.R-project.org/package=selection}

\hypertarget{ref-fife_estimating_2016}{}
Fife, D., Hunter, M. D., \& Mendoza, J. L. (2016). Estimating
Unattenuated Correlations With Limited Information About Selection
Variables: Alternatives to Case IV. \emph{Organizational Research
Methods}, \emph{19}(4), 593--615.
doi:\href{https://doi.org/10.1177/1094428115625323}{10.1177/1094428115625323}

\hypertarget{ref-graham_missing_2012}{}
Graham, J. W. (2012). \emph{Missing Data: Analysis and Design}. New
York, NY: Springer.

\hypertarget{ref-heckman_sample_1979}{}
Heckman, J. J. (1979). Sample Selection Bias as a Specification Error.
\emph{Econometrica}, \emph{47}(1), 153--161.
doi:\href{https://doi.org/10.2307/1912352}{10.2307/1912352}

\hypertarget{ref-kromrey_mean_1998}{}
Kromrey, J. D., \& Foster-Johnson, L. (1998). Mean Centering in
Moderated Multiple Regression: Much Ado about Nothing. \emph{Educational
and Psychological Measurement}, \emph{58}(1), 42--67.
doi:\href{https://doi.org/10.1177/0013164498058001005}{10.1177/0013164498058001005}

\hypertarget{ref-lawley_note_1944}{}
Lawley, D. N. (1944). A Note on Karl Pearson's Selection Formulae.
\emph{Proceedings of the Royal Society of Edinburgh. Section A.
Mathematical and Physical Sciences}, \emph{62}(01), 28--30.
doi:\href{https://doi.org/https://doi.org/10.1017/S0080454100006385}{https://doi.org/10.1017/S0080454100006385}

\hypertarget{ref-little_statistical_2014}{}
Little, R. J. A., \& Rubin, D. B. (2014). \emph{Statistical Analysis
with Missing Data}. Hoboken, NJ: John Wiley \& Sons.

\hypertarget{ref-rubin_inference_1976}{}
Rubin, D. B. (1976). Inference and missing data. \emph{Biometrika},
\emph{63}(3), 581--592.
doi:\href{https://doi.org/10.1093/biomet/63.3.581}{10.1093/biomet/63.3.581}

\hypertarget{ref-schafer_norm:_2010}{}
Schafer, J. L., Novo, A. A., \& Fox, J. (2010). \emph{Norm: Analysis of
multivariate normal datasets with missing values}. Retrieved from
\url{http://CRAN.R-project.org/package=norm}






\end{document}
